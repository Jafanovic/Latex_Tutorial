% https://tex.stackexchange.com/questions/132297/tcolorbox-newtcbtheorem-referencing
% https://tex.stackexchange.com/questions/121865/nameref-how-to-display-section-name-and-its-number

\documentclass{scrbook} 
\usepackage[most]{tcolorbox}
\newtcbtheorem[auto counter,number within=chapter]{mathexam}{Problem}{%
   % colback=purple!5,
   % colframe=blue!100!,
   fonttitle=\itshape,        % \bfseries,
   fontupper=\small,
   fontlower=\footnotesize,
   enforce breakable,         % breakable,
   compress page,
   enhanced,
   arc=1pt, left=0pt,right=0pt
}{mai}

\usepackage{nameref}
\usepackage[user, counter]{zref}
\usepackage{zref-xr}

\begin{document}
  \chapter{Algebra}\zlabel{chap:math1}
    This is a test for math.
    \begin{equation}
        E=mc^2 \zlabel{eq:1}
    \end{equation}
    This is a second test for math.
    \begin{equation}
        r = \sqrt{x^2 + y^2} \zlabel{eq:2}
    \end{equation}
    In the document A there is chapter \nameref{chap:math1} see the eqation \zref{eq:2}. Solve the
    problem shown in example \zref{exam038}
  
    \section{Bunch of examples}
      Let's solve following problems
      
      \begin{mathexam}{Gauss method}{exam038}
        \(3x + 4y = 5\)
      \end{mathexam}
\end{document}
% Macro definition \calcLength:
% https://www.latex4technics.com/?note=1D14
\documentclass[11pt]{scrartcl}
\usepackage{tikz,fp}
\usetikzlibrary{calc}

\makeatletter
\def\calcLength(#1,#2)#3{%
\pgfpointdiff{\pgfpointanchor{#1}{center}}%
             {\pgfpointanchor{#2}{center}}%
\pgf@xa=\pgf@x%
\pgf@ya=\pgf@y%
\FPeval\@temp@a{\pgfmath@tonumber{\pgf@xa}}%
\FPeval\@temp@b{\pgfmath@tonumber{\pgf@ya}}%
\FPeval\@temp@sum{(\@temp@a*\@temp@a+\@temp@b*\@temp@b)}%
\FProot{\FPMathLen}{\@temp@sum}{2}%
\FPround\FPMathLen\FPMathLen5\relax
\global\expandafter\edef\csname #3\endcsname{\FPMathLen}
}
\makeatother

\begin{document} 

  5cm = 5*28.45274 pt =142.2637pt

\begin{tikzpicture}
\coordinate (A) at (1,2);
\coordinate (B) at (4,6);
\calcLength(A,B){mylen}
% \draw (A) circle (\mylen pt); % pt is important here
\end{tikzpicture}
With calclength the length of AB is : \mylen

\begin{tikzpicture}
\coordinate (A) at (1,2);
\coordinate (B) at (4,6);
\path (A) let   \p1 = ($ (B) - (A) $),  \n1 = {veclen(\x1,\y1)} 
    in -- (B) node[draw]  {With veclen the length is :\n1};
\end{tikzpicture}   

\end{document}
\documentclass{scrbook}
% \usepackage{amsmath}
% \usepackage{hyperref} 

\usepackage{luakoma}
\usepackage{lualayout}
\usepackage{luamath} 


\begin{document}
%
% \input{../src/MAI/exam/exam040.tex}
Dosadíme-li získané hodnoty do třetí rovnice (dá to trochu práce), dostáváme obecnou  rovnici
roviny \(\varrho\)
\begin{subequations}\label{mai:eq041}
  \begin{equation}
    ax + by + cz + d= 0,  \label{mai:eq041a}
  \end{equation}
  \begin{equation}
    a = u_2v_3 - u_3v_2, \qquad 
    b = u_3v_1 - u_1v_3, \qquad 
    c = u_1v_2 - u_2v_1, \label{mai:eq041b}
  \end{equation}
  \begin{equation}
    d = (u_2v_3 - u_3v_2)x_A - 
        (u_3v_1 - u_1v_3)y_A - 
        (u_1v_2 - u_2v_1)z_A. \label{mai:eq041c}
  \end{equation}
\end{subequations}

\begin{equation}\label{fyz:eq252}
  \mathrm{grad}\,T =\nabla T = \left(\pder{T}{x},\, \pder{T}{y},\,\pder{T}{z} \right)
  \footnotemark
\end{equation}
\footnotetext{V naší symbolice představuje výraz \((a, b, c)\) vektor se složkami \(a\), \(b\),
\(c\). Použijeme-li jednotkové vektory $\vec{i},\,\vec{j},\,\vec{k}$, můžeme psát
$$\mathrm{grad}\,T=\nabla T = \vec{i}\pder{T}{x} + \vec{j}\pder{T}{y} + \vec{k}\pder{T}{z}$$}

\begin{figure}[ht!]  %\ref{fyz:fig133}
  \centering
  \caption{Standardní diagram barevnosti (chromatický diagram \textbf{CIE 
   1931}\protect\footnotemark) 
          (\cite[s.~468]{Feynman01})}
  \label{fyz:fig133}
\end{figure}
\footnotetext{Je jeden z prvních matematicky definovaných standardů}

\end{document}
% https://tex.stackexchange.com/questions/152358/animations-in-latex
\documentclass[preview,border={10pt 0pt 10pt 10pt}]{standalone}

% The version of LATEX released in Fall 2019 incorporates all of this package’s functionality (and
% more) into the LATEX kernel itself. As a result, there is no longer a need for the filecontents
% package. Please use the new, built-in filecontents environment instead.
\usepackage{filecontents}
    %% This is your file to be animated
\begin{filecontents*}{lsystems.tex}  
\documentclass{article}
\usepackage{tikz}
\usepackage[active,tightpage]{preview}\PreviewEnvironment{tikzpicture}
\usetikzlibrary{lindenmayersystems}
\pgfdeclarelindenmayersystem{A}{
\symbol{F}{\pgflsystemstep=0.6\pgflsystemstep\pgflsystemdrawforward}
\rule{A->F[+A][-A]}
}
\begin{document}
\foreach \n in {1,...,8} {
\begin{tikzpicture}[scale=10,rotate=90]
\draw (-.1,-.2) rectangle (.4,0.2);
\draw
    [blue,opacity=0.5,line width=0.1cm,line cap=round]
    l-system [l-system={A,axiom=A
    ,order=\n,angle=45,step=0.25cm}];
\end{tikzpicture}
}

\end{document}
\end{filecontents*}
%
\immediate\write18{pdflatex lsystems}

%% convert to GIF animation. Uncomment following line to have a gif animation in the same folder.
%\immediate\write18{convert -delay 10 -loop 0 -density 400 -alpha remove lsystems.pdf lsystems.gif}
%%

\usepackage{graphicx}
\usepackage{animate}
\begin{document}
\begin{preview}
%\animategraphics[controls,autoplay,loop,scale=<integer>]{<frame rate>}{<PDF filename without extension>}{<left blank>}{<left blank>}
\animategraphics[controls,autoplay,loop,scale=1]{2}{lsystems}{}{}
\end{preview}
\end{document}
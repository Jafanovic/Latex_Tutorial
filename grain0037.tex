% https://tex.stackexchange.com/questions/415369/tcolorbox-theorem-numbering
% See also page 105 in the tcolorbox manual. 
\documentclass{scrbook}
\usepackage{amsmath,amsthm,amssymb,parskip}
\usepackage[margin=1in]{geometry}
\usepackage{titling,multicol}
\usepackage[most]{tcolorbox}
\usepackage{csquotes}
\usepackage{graphicx,float}
\usepackage{hyperref}
\usepackage{amsmath}

\newtcbtheorem{mytheo}{Příklad}{colback=purple!5,colframe=blue!100!,fonttitle=\bfseries}{exam}

\newtcolorbox[
    auto counter,
    number within=section
  ]{pabox}[2][]{%
    colback=red!5!white,
    colframe=red!75!black,
    fonttitle=\bfseries,
    title=Příklad\smallskip~\thetcbcounter: #2,#1
  }


\begin{document}
\chapter{Kapitola 1}
\begin{mytheo*}{}
  text...
\end{mytheo*}

See \ref{exam:foo} 

\begin{mytheo}{Foo}{foo}
  text...
\end{mytheo}

\begin{mytheo}{Foo}{fooo}
  text...
\end{mytheo}

\section{Ahoj}
  \begin{pabox}[label={exam:001},nameref={Aplikace \(\int x^2\) tvaru}]{Hrátky s \Large\(\int x^2\)}
    This is a tcolorbox.
  \end{pabox}
    This box is my automatically numbered with \ref{exam:001} on page
    \pageref{exam:001}.
    The box is titled \enquote{\nameref{exam:001}}.

  
\end{document}
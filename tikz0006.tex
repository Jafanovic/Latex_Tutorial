\documentclass[12pt]{standalone}
\usepackage{pgfplots}
  \pgfplotsset{compat=newest}         % https://tex.stackexchange.com/questions/81899/
  \usetikzlibrary{arrows}
  \usetikzlibrary{calc}
  \usetikzlibrary{intersections}
  \usetikzlibrary{math}
  \usetikzlibrary{backgrounds,scopes} % https://tex.stackexchange.com/questions/159139/
                                      % (section 45, pages 579-582 in pgfmanual v 3.1.8b)

\usepackage{fp} % Fixed point arithmetic for TEX
\usepackage{mathrsfs}
\usepackage{siunitx}
% \usepackage{pgfmath}
\pagestyle{empty}


% Macro definition \calcLength:
% https://www.latex4technics.com/?note=1D14
\makeatletter
  \def\calcLength(#1,#2)#3{%
  \pgfpointdiff{\pgfpointanchor{#1}{center}}%
              {\pgfpointanchor{#2}{center}}%
  \pgf@xa=\pgf@x%
  \pgf@ya=\pgf@y%
  \FPeval\@temp@a{\pgfmath@tonumber{\pgf@xa}}%
  \FPeval\@temp@b{\pgfmath@tonumber{\pgf@ya}}%
  \FPeval\@temp@sum{(\@temp@a*\@temp@a+\@temp@b*\@temp@b)}%
  \FProot{\FPMathLen}{\@temp@sum}{2}%
  \FPround\FPMathLen\FPMathLen5\relax
  \global\expandafter\edef\csname #3\endcsname{\FPMathLen}
  }
\makeatother

% https://tex.stackexchange.com/questions/99550/
\newcommand\pttomm[1]{%
  \pgfmathsetmacro\len{#1pt/1mm}Here, with the current font, #1\,mm is \len{}\,mm.
}

% ~~~~~~~~~~~ XY Coordinate ~~~~~~~~~~~~~~~~~~~~~
% tikz0008.tex
\makeatletter
\def\extractcoord#1#2#3{
  \path let \p1=(#3) in \pgfextra{
    \pgfmathsetmacro#1{\x{1}/\pgf@xx}
    \pgfmathsetmacro#2{\y{1}/\pgf@yy}
    \xdef#1{#1} \xdef#2{#2}
  };
}

\begin{document}
  \definecolor{my_green}{rgb}{0,0.39,0}
  \definecolor{my_red}{rgb}{1,0,0}
  \definecolor{my_azureblue}{rgb}{0,1,1}
  \definecolor{my_black}{rgb}{0.266,0.266,0.266}
  \definecolor{my_darkblue}{rgb}{0.30,0.30,1}
  % poloměr kružnice k_S
  \def\Rs{2.2}


  \begin{tikzpicture}[
      line cap=round, line join=round, >=triangle 45, x=10mm, y=10mm,
      background grid/.style={thick,draw=my_black!10,step=1}, show background grid
    ]
    % \draw(-6.841530775492609,-1) rectangle (13.068148089335633,23);

    \coordinate (bodS) at (0,9);
    \coordinate (bodA) at (7,9);
     \extractcoord\xA\yA{bodA}
    \coordinate (bodL) at ($(bodS)!0.5!(bodA)$);
      \calcLength(bodS,bodL){lengthLS}
      \calcLength(bodS,bodA){lengthSA}


    % poloměr kružnice k_S
    \draw [line width=4pt, name path=Circ_stredS] (bodS) circle (\Rs);
    \draw [line width=2pt] (bodS) 
      node[font=\boldmath, anchor=north east
        ] {\(S\)} -- (bodA) 
      node[font=\boldmath, anchor=north west
        ] {\(A\)};
    

% Zkonstruujme střed úsečky SA, který označíme L
% https://tex.stackexchange.com/questions/66216/draw-arc-in-tikz-when-center-of-circle-is-specified
% pomocné kružnice z bodu S pro výpočet polohy bodu L, ačkoliv už ho známe
    \begin{scope}[on background layer]
      \draw [shift={(bodS)},line width=2pt,
            dash pattern=on 3pt off 4pt,
            color=my_azureblue,
            name path=auxcircHP
        ]  
        plot[domain=0.0:1.3,variable=\t]
          ({1*(\lengthSA/1pt)*cos(\t r) + 0*(\lengthSA/1pt)*sin(\t r)},
           {0*(\lengthSA/1pt)*cos(\t r) + 1*(\lengthSA/1pt)*sin(\t r)});
      
      \draw [shift={(bodA)},line width=2pt,
            dash pattern=on 3pt off 4pt,
            color=my_azureblue,
            name path=auxcircHL
        ]  
        plot[domain=1.8:3.14,variable=\t]
          ({1*(\lengthSA/1pt)*cos(\t r) + 0*(\lengthSA/1pt)*sin(\t r)},
           {0*(\lengthSA/1pt)*cos(\t r) + 1*(\lengthSA/1pt)*sin(\t r)});
                                        
      \draw [shift={(bodS)},line width=2pt,
            dash pattern=on 3pt off 4pt,
            color=my_azureblue,
            name path=auxcircLP
        ]  
        plot[domain=5.0:6.28,variable=\t]
          ({1*(\lengthSA/1pt)*cos(\t r) + 0*(\lengthSA/1pt)*sin(\t r)},
           {0*(\lengthSA/1pt)*cos(\t r) + 1*(\lengthSA/1pt)*sin(\t r)});
      
      \draw [shift={(bodA)}, line width=2pt,
            dash pattern=on 3pt off 4pt,
            color=my_azureblue,
            name path=auxcircLL
        ]  
        plot[domain=3.14:4.5,variable=\t]
          ({1*(\lengthSA/1pt)*cos(\t r) + 0*(\lengthSA/1pt)*sin(\t r)},
           {0*(\lengthSA/1pt)*cos(\t r) + 1*(\lengthSA/1pt)*sin(\t r)});

      % intersection points - modré pomocné kružnice
      \path[name intersections={of=auxcircHP and auxcircHL,by={a}}];
      \path[name intersections={of=auxcircLP and auxcircLL,by={b}}];
      
      \draw [line width=2pt,
             dash pattern=on 3pt off 4pt,
             color=my_azureblue
        ] (a)-- (b);
      
      \draw [-stealth,line width=2pt, color = my_black!50] 
        (bodS) -- +(150:\Rs) node [midway, below] {\(R_S\)}; % polar coordinates

      % https://tex.stackexchange.com/questions/38473/how-can-i-compute-the-distance-between-two-coordinates-in-tikz
      \draw [-stealth,line width=2pt,color=my_red!50] 
        let \p1 = ($(bodA)-(bodL)$) in (bodL) 
          -- ++(300:{veclen(\x1,\y1)}) node[midway, left] {\(R_{LA}\)};

      \draw [-stealth,line width=2pt,color=my_azureblue!50] 
        let \p1 = ($(bodA)-(bodS)$) in (bodA) 
          -- ++(130:{veclen(\x1,\y1)}) node[midway, left] {\(R_{SA}\)};

      \draw [color=my_red] 
        let \p1 = ($(bodA)-(bodL)$) in (bodL) ++(300:{veclen(\x1,\y1)}) node[below] {\(k_L\)};
    \end{scope}
      
      % Narýsujme kružnici k_{L} se středem v bodě L o poloměru R_{L}, kde poloměr R_{L} je roven 
      % velikosti úsečky LA (a také LS).
      \draw [line width=2pt,color=my_red, 
      name path=Circ_stredL, dash pattern=on 3pt off 4pt
      ] (bodL) circle (\lengthLS pt);
      
      % intersection points - modré červené a černé kružnice
      \path[name intersections={of=Circ_stredS and Circ_stredL, name=cross}];
      \coordinate (TA) at (cross-1);
      \coordinate (TB) at (cross-2); 
      
      \draw [color=my_black] 
        let \p1 = ($(bodS)-(TA)$) in (bodS) ++(150:{veclen(\x1,\y1)}) node[left] {\(k_S\)};
      
  
    % Body T_{1} a A veďme přímku, která je tečnou t_{1} ke kružnici k_{S} v bodě T_{1}
    \extractcoord\xTA\yTA{TA}
    \def\k{((\yTA-\yA)/(\xTA-\xA))}
    \def\q{(\yTA-\k*\xTA)}
    \draw [line width=2pt,domain=-3:8] plot(\x,{(\q + \k*\x)});
    
    % Analogicky zkonstruujme tečnu {\displaystyle t_{2}}t_{2}.
    \extractcoord\xTB\yTB{TB}
    \def\k{((\yTB-\yA)/(\xTB-\xA))}
      \def\q{(\yTB-\k*\xTB)}
      \draw [line width=2pt,domain=-3:8] plot(\x,{(\q + \k*\x)});
      
      \draw [line width=2pt] (bodS) -- (TA);
      
      % značka úhlu -stealth metoda kreslení oblouku pomocí "arc" jsem nahradil užitím metody clip a 
      % kreslením kruhu. Ten se nakreslí jen ve vnitřku ohraničeném funkcí "clip"    
      \begin{scope}    
        \path[clip] (bodS) -- (TA) -- (bodA) -- cycle;
        \node[circle,draw,minimum size=20mm, 
                line width=2pt,color=my_green,
                fill=my_green,fill opacity=0.1] 
                at (TA) (circ) {};
        % \draw [shift={(TA)},line width=2pt,color=my_green,fill=my_green,fill opacity=0.1] 
        %   (0,0) -- (-115.52125021455065:0.7) arc (-115.52125021455065:-25.521250214550598:0.7) -- cycle;
        \draw[color=my_green] (TA) node[xshift=0.4cm, yshift=-1.3cm] {\(\alpha = \ang{90}\)};
      \end{scope}
      
    \begin{scope}
      \draw [fill=my_darkblue] (bodA) circle (3pt);
      \draw [fill=my_darkblue] (bodS) circle (3pt);
      \draw [fill=my_black]    (bodL) circle (3pt) node [anchor=north east] {$L$};
      
      % \draw [color=my_darkblue](-4,21) node[anchor=north west] 
      %   {\textbf{Narýsování tečny  procházející  bodem  podle Thaletovy věty}};
      % \draw (-3,20) node[anchor=north west] 
      %   {\parbox{15 cm}{Nechť je kružnice \(k_{S}\) se středem \(S\) a poloměrem \(R_{S}\) a 
      %     bod A vně této kružnice. 
      %     \\ Ukážeme konstukci tečny ke kružnici, která prochází bodem \(A\).}};
      
      % \draw (-3,1) node[anchor=north west] 
      %   {Konstrukce tečny ke kružnici \(k_S\) procházející daným bodem \(A\).};
  
      \draw[fill=my_black]  (TA) circle (3pt);
      \draw[fill=my_black]  (TB) circle (3pt);
      \draw[color=my_black] (TA) node[above] {$T_1$};
      \draw[color=my_black] (TB) node[below] {$T_2$};
      \node[xshift=-20pt] at ($ (bodA)!.25!(TB) $) {\(t_2\)};
      \node[xshift=-20pt] at ($ (bodA)!.25!(TA) $) {\(t_1\)};
      

      \node[xshift=-20pt] at ($ (bodA)!.25!(TB) $) {\(t_2\)};
    \end{scope}  
  \end{tikzpicture}
  % debuging
  % With calclength the length of AB is : \lengthSA conversion \pttomm{\lengthSA}  \newline
  % TA: $(\pgfmathprintnumber[precision=4]{\xTA},\pgfmathprintnumber[precision=4]{\yTA})$\par
  % TB: $(\pgfmathprintnumber[precision=4]{\xTB},\pgfmathprintnumber[precision=4]{\yTB})$\par
  \end{document}
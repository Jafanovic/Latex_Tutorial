\input{../Luaking/tex/ebook}
\documentclass{scrbook}

\usepackage{amsmath}  
\usepackage{mdframed}
\usepackage{blindtext}

\mdfdefinestyle{mdeq}{
  % outermargin    =  1cm,
  skipabove        =  \topskip,
  skipbelow        =  \topskip,
  innertopmargin  =  4pt,
  innerbottommargin =  4pt,
  roundcorner      =  20pt
  }

\providecommand\eBookMode{true}
\begin{document}
\chapter{Chapter test}
\section{Motivace}
  Problém \emph{neurčitého integrálu}, neboli \textbf{primitivní funkce}, lze vyložit velmi 
  jednoduše: Máme podezření, že zadaná funkce \(f(x)\) vznikla derivováním jisté, zatím neznámé, 
  funkce \(F(x)\). Dokážeme ji najít? 
  \begin{mdframed}[style=mdeq] 
    \begin{equation}
      \int x^2 dx
    \end{equation}
  \end{mdframed}
  K danému problému můžeme přistupovat také fyzikálně: Zavedením pojmu derivace funkce jsme 
  \begin{mdframed}[style=mdeq]
    \begin{equation}\label{mai:eq102}
      \boxed{\int f(x) dx = F(x) + c}
    \end{equation}
  \end{mdframed}
  motivovali důležitým požadavkem definovat okamžitou rychlost pohybu bodu po přímce. Existuje 
  přirozeně i požadavek opačný, tj. nalézt zákon dráhy pohybu bodu po přímce, je-li dána jeho 
  okamžitá rychlost jako funkce času \cite[s.~253]{Brabec1989}. Vše si ukážeme na následujícím 
  příkladu:     
\end{document}
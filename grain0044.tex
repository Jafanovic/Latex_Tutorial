% https://tex.stackexchange.com/questions/301993/create-custom-note-environment-with-tcolorbox
\documentclass[11pt]{article}
\usepackage{amsmath,amssymb,mathtools}
\usepackage{diffcoeff}
\usepackage{nccmath}
\usepackage{scalerel}

\usepackage[most]{tcolorbox}

\newtcolorbox{note}[1][]{%
  enhanced jigsaw, % better frame drawing
  borderline west={2pt}{0pt}{red}, % straight vertical line at the left edge
  sharp corners, % No rounded corners
  boxrule=0pt, % no real frame,
  fonttitle={\large\bfseries},
  coltitle={black},  % Black colour for title
  title={Note:\ },  % Fixed title
  attach title to upper, % Move the title into the box
  #1
}

\usepackage{blindtext}

\begin{document}

  \blindtext
  \begin{note}
      Here comes a useful note for demonstration. Let's add enough text to break the line, just to
      be sure the vertical line extends with the text.
  \end{note}

  \begin{note}[borderline west={2pt}{0pt}{blue}, colback=yellow, drop shadow={red,opacity=0.6}]
      Here comes a useful note for demonstration. Let's add enough text to break the line, just to
      be sure the vertical line extends with the text.
  \end{note}

  \begin{fleqn}[\parindent] 
    \begin{equation}
      \int(u(x)v(x))'{x} = u(x)v(x) + c  =  \\
      = \int\left(u'(x)v(x)+u(x)v'(x)\right){x}.
    \end{equation}
  \end{fleqn}

  \begin{equation*}
  \begin{multlined}
    S = \scalerel{\int_{t_1}^{t_2}}{
      \left[
          \frac{m}{2}\left(\diff{\underline{x}}{t}\right)^2-V(\underline{x}) 
           + m\diff{x}{t}\diff{\eta}{t} - \eta V'(\underline{x}) 
      \right.} \\
      % \left.
          + (\text{druhý řád a vyšší řády})
      \Bigg]
    dt  \stackrel{?}{=} = 0.
  \end{multlined}
  \end{equation*}
  
  \blindtext
  \begin{equation*}
    KE = \dfrac{m}{2}\left[\diff[2]{x}{t} +\diff[2]{y}{t} + \diff[2]{z}{t}\right].
  \end{equation*}

  \(\mathcal{L}\)
\end{document}
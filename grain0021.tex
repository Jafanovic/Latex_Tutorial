% https://tex.stackexchange.com/questions/516093/replacement-minitoc-package-style-by-koma-script-possibilities
% https://komascript.de/node/2107
\documentclass{scrreprt}
 
\usepackage{hyperref}
\usepackage{mwe}
 
\makeatletter
\ifundefinedorrelax{ext@toc}{%
  \newcommand*{\ext@toc}{toc}
  \renewcommand{\addtocentrydefault}[3]{%
    \expandafter\tocbasic@addxcontentsline\expandafter{\ext@toc}{#1}{#2}{#3}%
  }
}{}
\newcommand*{\switchcontents}[1]{%
  \edef\ext@toc{toc#1}% define new toc extension
  % If you do not want bookmarks for the following \chapter etc. remove
  % the next three lines.
  \ifundefinedorrelax{hypersetup}{}{%
    \edef\reserved@a{\noexpand\hypersetup{bookmarkstype=\ext@toc}}\reserved@a
  }%
  % Reserve the toc extension
  \ifattoclist{\ext@toc}{}{\addtotoclist[toc]{\ext@toc}}%
}
\newcommand*{\partcontents}[1][\thepart]{%
  \ifattoclist{toc#1}{%
    \switchcontents{#1}%
  }{%
    \switchcontents{#1}%
    \listoftoc[Contents of this Part]{\ext@toc}% show contents only for new extensions
  }%
}

\newcommand*{\chaptercontents}[1][\thechapter]{%
  \ifattoclist{toc#1}{%
    \switchcontents{#1}%
  }{%
    \switchcontents{#1}%
    \listoftoc[Contents of this Chapter]{\ext@toc}% show contents only for new extensions
  }%
}

\makeatother
 
\begin{document}
\tableofcontents
\part{First Part}
\blinddocument
 
\part{Second Part}
\partcontents% use \jobname.toc\thepart as new toc file and show the contents
\blinddocument
 
\switchcontents{}% switch back to \jobname.toc as toc file
% alternative:
%\partcontents[]% switch back to \jobname.toc as toc file
\part{Third Part}
\blinddocument
 
\part{Fourth Part}
\partcontents% use \jobname.toc\thepart as new toc file and show the contents
\blinddocument

\chapter {New chapter}
\chaptercontents
  \section{New section}
  \section{Test section}  
  \begin{figure}
    \includegraphics[width=.48\linewidth]{example-image-a} \hfill
    \includegraphics[width=.48\linewidth]{example-image-b}
    \caption{MWE to demonstrate how to place to images side-by-side}
  \end{figure}
  \blindtext
 
\end{document}
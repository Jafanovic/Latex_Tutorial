% https://tex.stackexchange.com/questions/520667/fonstspec-package-creates-character-consists-of-the-seperate-symbol-and-accent
% https://tex.stackexchange.com/questions/468795/best-way-of-using-the-full-range-of-full-fonts-styles-faces-for-latin-modern-rom
% https://tex.stackexchange.com/questions/186796/neo-euler-wont-display-greek-letters?noredirect=1&lq=1
\documentclass[
    fontsize=10pt,
    twocolumn,
    open=any, 
    titlepage=false, 
    chapterprefix=true
]{scrbook}

\usepackage[
    % showframe=true,
    verbose=true,                   % displays the parameter results on the terminal
    paperwidth  = 158mm,            % 
    paperheight = 210mm,            %
    % landscape,                    %
    includehead, includefoot,       % include the head and foot into total body
    scale=1.0,                      %
    nomarginpar,                    % shrinks spaces for marginal notes to 0pt,
    left=3pt, right=3pt,            %
    top=2pt, bottom=9pt,            %
    headsep=2pt,                    %
    headheight=12pt,                % At least 11.0pt needed!! 
    showframe=false, nofoot,        % 
    % noheadfoot=true,              %  
    heightrounded,                  % better use it
]{geometry} 

% Additional Options
    \usepackage{mathtools}
    \usepackage{unicode-math}
    \usepackage{fontspec}
    %fontspec options: 
    \setmainfont{TeX Gyre Pagella}[
        Scale=1.0,
        Ligatures={Common, Discretionary, TeX}]
    \setmathfont[math-style=upright]{Neo Euler}

\begin{document}
\setlength\parindent{0pt}
    \setmainfont{TeX Gyre Adventor}
    \section{TEX Gyre Adventor}
    Příliš žluťoučký kůň úpěl ďábelské ódy! \newline

    Pokračujme v procesu zobecňování. Existují nějaké další rovnice, které neumíme vyřešit? Ano,
    existují. Například nelze vyřešit rovnici \(b=2^\frac{1}{2} =\sqrt{2}\). Není možné najít takové
    racionální číslo (zlomek), jehož druhá mocnina je rovna 2. V moderní době je na tuto otázku
    velmi snadná odpověď'. Známe desetinný systém, takže nám nedělají problémy nekonečná desetinná
    čísla jako způsob aproximace druhé odmocniny ze 2. Historicky to však byl velký problém pro
    staré Řeky. Skutečně přesná definice toho, o co tu jde, vyžaduje znát něco z podstaty spojitosti
    a uspořádanosti, a to je na tomto místě právě nejtěžší krok v procesu zobecňování. Formálně a
    velmi přesně to provedl. Avšak, bez obav o matematickou přesnost, můžeme snadno
    pochopit, co zamýšlíme provést, tj. aproximovat takové číslo celou posloupností desetinných
    zlomků (neboť každé konečné desetinné číslo je racionální číslo), která se stále více a více
    přibližuje k hledanému výsledku. To nám pro naše účely postačí, umožní nám to manipulovat s
    iracionálními čísly i jejich výpočet (jako například čísla \(\sqrt{2}\)) s libovolnou přesností.
    \newpage

    \setmainfont{TeX Gyre Bonum}
    \section{TeX Gyre Bonum}
    Příliš žluťoučký kůň úpěl ďábelské ódy!  \newline

    Pokračujme v procesu zobecňování. Existují nějaké další rovnice, které neumíme vyřešit? Ano,
    existují. Například nelze vyřešit rovnici \(b=2^\frac{1}{2} =\sqrt{2}\). Není možné najít takové
    racionální číslo (zlomek), jehož druhá mocnina je rovna 2. V moderní době je na tuto otázku
    velmi snadná odpověď'. Známe desetinný systém, takže nám nedělají problémy nekonečná desetinná
    čísla jako způsob aproximace druhé odmocniny ze 2. Historicky to však byl velký problém pro
    staré Řeky. Skutečně přesná definice toho, o co tu jde, vyžaduje znát něco z podstaty spojitosti
    a uspořádanosti, a to je na tomto místě právě nejtěžší krok v procesu zobecňování. Formálně a
    velmi přesně to provedl. Avšak, bez obav o matematickou přesnost, můžeme snadno
    pochopit, co zamýšlíme provést, tj. aproximovat takové číslo celou posloupností desetinných
    zlomků (neboť každé konečné desetinné číslo je racionální číslo), která se stále více a více
    přibližuje k hledanému výsledku. To nám pro naše účely postačí, umožní nám to manipulovat s
    iracionálními čísly i jejich výpočet (jako například čísla \(\sqrt{2}\)) s libovolnou přesností.
    \newpage

    \setmainfont{TeX Gyre Chorus}
    \section{TeX Gyre Chorus}
    Příliš žluťoučký kůň úpěl ďábelské ódy!  \newline

    Pokračujme v procesu zobecňování. Existují nějaké další rovnice, které neumíme vyřešit? Ano,
    existují. Například nelze vyřešit rovnici \(b=2^\frac{1}{2} =\sqrt{2}\). Není možné najít takové
    racionální číslo (zlomek), jehož druhá mocnina je rovna 2. V moderní době je na tuto otázku
    velmi snadná odpověď'. Známe desetinný systém, takže nám nedělají problémy nekonečná desetinná
    čísla jako způsob aproximace druhé odmocniny ze 2. Historicky to však byl velký problém pro
    staré Řeky. Skutečně přesná definice toho, o co tu jde, vyžaduje znát něco z podstaty spojitosti
    a uspořádanosti, a to je na tomto místě právě nejtěžší krok v procesu zobecňování. Formálně a
    velmi přesně to provedl. Avšak, bez obav o matematickou přesnost, můžeme snadno
    pochopit, co zamýšlíme provést, tj. aproximovat takové číslo celou posloupností desetinných
    zlomků (neboť každé konečné desetinné číslo je racionální číslo), která se stále více a více
    přibližuje k hledanému výsledku. To nám pro naše účely postačí, umožní nám to manipulovat s
    iracionálními čísly i jejich výpočet (jako například čísla \(\sqrt{2}\)) s libovolnou přesností.
    \newpage

    \setmainfont{TeX Gyre Heros}
    \section{TeX Gyre Heros}
    Příliš žluťoučký kůň úpěl ďábelské ódy!  \newline

    Pokračujme v procesu zobecňování. Existují nějaké další rovnice, které neumíme vyřešit? Ano,
    existují. Například nelze vyřešit rovnici \(b=2^\frac{1}{2} =\sqrt{2}\). Není možné najít takové
    racionální číslo (zlomek), jehož druhá mocnina je rovna 2. V moderní době je na tuto otázku
    velmi snadná odpověď'. Známe desetinný systém, takže nám nedělají problémy nekonečná desetinná
    čísla jako způsob aproximace druhé odmocniny ze 2. Historicky to však byl velký problém pro
    staré Řeky. Skutečně přesná definice toho, o co tu jde, vyžaduje znát něco z podstaty spojitosti
    a uspořádanosti, a to je na tomto místě právě nejtěžší krok v procesu zobecňování. Formálně a
    velmi přesně to provedl. Avšak, bez obav o matematickou přesnost, můžeme snadno
    pochopit, co zamýšlíme provést, tj. aproximovat takové číslo celou posloupností desetinných
    zlomků (neboť každé konečné desetinné číslo je racionální číslo), která se stále více a více
    přibližuje k hledanému výsledku. To nám pro naše účely postačí, umožní nám to manipulovat s
    iracionálními čísly i jejich výpočet (jako například čísla \(\sqrt{2}\)) s libovolnou přesností.
    \newpage

    \setmainfont{TeX Gyre Pagella}
    \section{TeX Gyre Pagella}
    Příliš žluťoučký kůň úpěl ďábelské ódy!  \newline

    Pokračujme v procesu zobecňování. Existují nějaké další rovnice, které neumíme vyřešit? Ano,
    existují. Například nelze vyřešit rovnici \(b=2^\frac{1}{2} =\sqrt{2}\). Není možné najít takové
    racionální číslo (zlomek), jehož druhá mocnina je rovna 2. V moderní době je na tuto otázku
    velmi snadná odpověď'. Známe desetinný systém, takže nám nedělají problémy nekonečná desetinná
    čísla jako způsob aproximace druhé odmocniny ze 2. Historicky to však byl velký problém pro
    staré Řeky. Skutečně přesná definice toho, o co tu jde, vyžaduje znát něco z podstaty spojitosti
    a uspořádanosti, a to je na tomto místě právě nejtěžší krok v procesu zobecňování. Formálně a
    velmi přesně to provedl. Avšak, bez obav o matematickou přesnost, můžeme snadno
    pochopit, co zamýšlíme provést, tj. aproximovat takové číslo celou posloupností desetinných
    zlomků (neboť každé konečné desetinné číslo je racionální číslo), která se stále více a více
    přibližuje k hledanému výsledku. To nám pro naše účely postačí, umožní nám to manipulovat s
    iracionálními čísly i jejich výpočet (jako například čísla \(\sqrt{2}\)) s libovolnou přesností.
    \newpage

    \setmainfont{TeX Gyre Schola}
    \section{TeX Gyre Schola}
    Příliš žluťoučký kůň úpěl ďábelské ódy!  \newline

    Pokračujme v procesu zobecňování. Existují nějaké další rovnice, které neumíme vyřešit? Ano,
    existují. Například nelze vyřešit rovnici \(b=2^\frac{1}{2} =\sqrt{2}\). Není možné najít takové
    racionální číslo (zlomek), jehož druhá mocnina je rovna 2. V moderní době je na tuto otázku
    velmi snadná odpověď'. Známe desetinný systém, takže nám nedělají problémy nekonečná desetinná
    čísla jako způsob aproximace druhé odmocniny ze 2. Historicky to však byl velký problém pro
    staré Řeky. Skutečně přesná definice toho, o co tu jde, vyžaduje znát něco z podstaty spojitosti
    a uspořádanosti, a to je na tomto místě právě nejtěžší krok v procesu zobecňování. Formálně a
    velmi přesně to provedl. Avšak, bez obav o matematickou přesnost, můžeme snadno
    pochopit, co zamýšlíme provést, tj. aproximovat takové číslo celou posloupností desetinných
    zlomků (neboť každé konečné desetinné číslo je racionální číslo), která se stále více a více
    přibližuje k hledanému výsledku. To nám pro naše účely postačí, umožní nám to manipulovat s
    iracionálními čísly i jejich výpočet (jako například čísla \(\sqrt{2}\)) s libovolnou přesností.
    \newpage

    \setmainfont{TeX Gyre Termes}
    \section{TeX Gyre Termes}
    Příliš žluťoučký kůň úpěl ďábelské ódy!  \newline

    Pokračujme v procesu zobecňování. Existují nějaké další rovnice, které neumíme vyřešit? Ano,
    existují. Například nelze vyřešit rovnici \(b=2^\frac{1}{2} =\sqrt{2}\). Není možné najít takové
    racionální číslo (zlomek), jehož druhá mocnina je rovna 2. V moderní době je na tuto otázku
    velmi snadná odpověď'. Známe desetinný systém, takže nám nedělají problémy nekonečná desetinná
    čísla jako způsob aproximace druhé odmocniny ze 2. Historicky to však byl velký problém pro
    staré Řeky. Skutečně přesná definice toho, o co tu jde, vyžaduje znát něco z podstaty spojitosti
    a uspořádanosti, a to je na tomto místě právě nejtěžší krok v procesu zobecňování. Formálně a
    velmi přesně to provedl. Avšak, bez obav o matematickou přesnost, můžeme snadno
    pochopit, co zamýšlíme provést, tj. aproximovat takové číslo celou posloupností desetinných
    zlomků (neboť každé konečné desetinné číslo je racionální číslo), která se stále více a více
    přibližuje k hledanému výsledku. To nám pro naše účely postačí, umožní nám to manipulovat s
    iracionálními čísly i jejich výpočet (jako například čísla \(\sqrt{2}\)) s libovolnou přesností.
\end{document}
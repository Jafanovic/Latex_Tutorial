% https://tex.stackexchange.com/questions/415369/tcolorbox-theorem-numbering
\documentclass[twocolumn]{scrbook}
\usepackage{amsmath,amsthm,amssymb,parskip}
% \usepackage{lualayout}
% \usepackage{luamath}
\usepackage{nccmath}
% https://tex.stackexchange.com/questions/444153/why-does-not-shortintertext-command-work-if-we-load-nccmath-after-mathtool
\usepackage{mathtools}     
  % https://tex.stackexchange.com/questions/84302/
  \newcommand{\dd}[1]{\mathop{}\!\mathrm{d}#1}  

\usepackage[showframe]{geometry}
\usepackage{titling,multicol}
\usepackage[most]{tcolorbox}

% \usepackage{graphicx,float}
\usepackage{bigints}
\usepackage{hyperref}
\usepackage{blindtext}

\newtcbtheorem[auto counter,number within=chapter]{mathexam}{Příklad}{%
    colback=green!25,
    colframe=blue!70!,
    coltitle=yellow,
    lefttitle=2em,
    fonttitle=\large\itshape, % \bfseries
    before=\par\medskip\noindent,
    parbox=false,
    breakable,
    enhanced,
    arc=1pt,
    left=0pt,
    right=0pt
  }{mai}


\newcounter{example}

\colorlet{colexam}{red!75!black}
\newtcolorbox[use counter=example]{myexample}{%
    empty,
    title={Example \thetcbcounter},
    attach boxed title to top left,
    boxed title style={empty,size=minimal,toprule=2pt,top=4pt,
    overlay={\draw[colexam,line width=2pt]
      ([yshift=-1pt]frame.north west)--([yshift=-1pt]frame.north east);}},
    coltitle=colexam,
    fonttitle=\Large\bfseries,
    before=\par\medskip\noindent,
    parbox=false,
    boxsep=0pt,
    left=0pt,
    right=3mm,
    top=4pt,
    breakable,
    pad at break*=0mm,
    vfill before first,
    overlay unbroken={\draw[colexam,line width=1pt]
        ([yshift=-1pt]title.north east)--([xshift=-0.5pt,yshift=-1pt]title.north-|frame.east)
      --([xshift=-0.5pt]frame.south east)--(frame.south west); },
    overlay first={\draw[colexam,line width=1pt]
      ([yshift=-1pt]title.north east)--([xshift=-0.5pt,yshift=-1pt]title.north-|frame.east)
      --([xshift=-0.5pt]frame.south east); },
    overlay middle={\draw[colexam,line width=1pt] ([xshift=-0.5pt]frame.north east)
      --([xshift=-0.5pt]frame.south east); },
    overlay last={\draw[colexam,line width=1pt] ([xshift=-0.5pt]frame.north east)
      --([xshift=-0.5pt]frame.south east)--(frame.south west);},%
  }

  
\begin{document}
\chapter{První kapitola}
\section{Moje sekce 1}
  \blindtext 
  
  See \ref{mai:exam126}

  \begin{mathexam}{\(\bigint\left(7x^2+2x^3+5\cos x\right)\dd{x}\)}{exam126}
    V tomto případě \(f_1(x) = x^5\), \(f_2(x) = x^3\), \(f_3(x) = 5\cos x\), \(c_1=7\), \(c_1=2\),
    \(c_3=5\). Protože v intervalu \((-\infty, \infty)\) existují podle
    \eqref{mai:IchapVIIsecIIssecI} integrály
    \begin{align*}
      \int x^5\dd{x}   &= \frac{x^6}{6} + C_1, \\
      \int x^3\dd{x}   &= \frac{x^4}{4} + C_2, \\
      \int\cos x\dd{x} &= \sin x +C_3,
    \end{align*}
    platí podle věty \eqref{mai:lemma013} 
    \begin{fleqn}[0pt]
      \begin{multline*}
        \int(7x^2+2x^3+5\cos x)\dd{x} = \\
          = 7\int x^5\dd{x} + 2\int x^3\dd{x} + 5\int\cos x\dd{x},
      \end{multline*}
    \end{fleqn}
      a tedy (integrační konstanty shneme v jedinou)
      \begin{fleqn}[0pt]
        \begin{multline*}
          \int(7x^2+2x^3+5\cos x)\dd{x} = \\
            = \frac{7}{6}x^6 + \frac{1}{2}x^4 + 5\sin x + c.
        \end{multline*}
      \end{fleqn}
      a tedy (integrační konstanty shneme v jedinou)
      \begin{fleqn}[0pt]
        \begin{multline*}
          \int(7x^2+2x^3+5\cos x)\dd{x} = \\
            = \frac{7}{6}x^6 + \frac{1}{2}x^4 + 5\sin x + c.
        \end{multline*}
      \end{fleqn}
  \end{mathexam}
  \blindtext

  \begin{mathexam}{Je dána okamžitá rychlost $v$ pohybu bodu po přímce (ose) $x$ rovnicí $v(t) = 2t + 1$, 
    $t\in\langle -\infty,+\infty \rangle$. Najděte zákon dráhy pohybu, je-li známo, že v čase $t 
    = 0$ měl bod polohu $x = x_0$ \cite[p.~253]{Brabec1989}.}{exam117}

    Označíme-li $x(t)$ polohu bodu v okamžiku $t$, pak $v(t) = \frac{dx}{dt}$. Hledáme tedy 
    funkci $x = x(t)$, pro níž platí 
    \begin{align}
      \frac{dx}{dt} &= 2t + 1 \qquad x(0) = x_0.  \nonumber \\ 
      \shortintertext{Je ihned patrné, že první podmínce vyhovuje nekonečně mnoho funkcí}
      x(t)          &= t^2 + t + C,           \label{MA:int_ex_09}    \\ 
      \shortintertext{ kde $C$ je libovolná konstanta. Funkce, která splňuje i druhou podmínku 
        (říkáme ji též počáteční podmínka), najdeme z rovnice \ref{MA:int_ex_09} dosazením dané 
        podmínky $t = 0,\ x = x_0$. Dostaneme $x_0 = C$. Dosazením do \ref{MA:int_ex_09} za $C$ 
        plyne hledaný zákon dráhy}
      x(t)          &= t^2+t+x_0.                 
    \end{align}
    Jednoduchou zkouškou se přesvědčíme, že tato funkce splňuje obě dané podmínky a zároveň
    vidíme, že hledaná primitivní funkce daných vlastností je jediná.
\end{mathexam}

\begin{myexample}
  \blindtext[3]
\end{myexample}

  
\end{document}
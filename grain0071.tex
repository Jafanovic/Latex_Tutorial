% Drawing Realistic Linux Command Shell Windows with tcolorbox
% https://tex.stackexchange.com/questions/517976/drawing-realistic-linux-command-shell-windows-with-tcolorbox
\documentclass[]{scrbook}
  % \usepackage[a4paper,margin=1in]{geometry}
  % \usepackage{tcolorbox,url,tikzlings}
  % \tcbuselibrary{skins,xparse,listings}

  % \usepackage{luamath}
  \usepackage{luaCodeLst}

\begin{document}

% \newtcblisting{ubuntu}{%
%   colback=violet!50!black,
%   colupper=white,
%   colframe=gray!65!black,
%   listing only,
%   width=7cm,
%   left=1pt,
%   right=1pt,
%   top=0pt,
%   listing options={
%     % basicstyle=\small\ttfamily\bfseries, 
%     style=tcblatex,
%     language=sh,
%     escapeinside=``,
%     aboveskip=0pt,belowskip=0pt,
%     extendedchars=true
%   },
%   title={\textcolor{orange}{\Huge{$\bullet$}}{\textcolor{gray}{\Huge{$\bullet\bullet$}}}},
%   every listing line={\MyUbuntuPrompt},
%   enhanced,
%   overlay={ \begin{tcbinvclipframe}
%               % \penguin[shift={([xshift=2cm,yshift=-0.8cm]frame.north)}]
%               \penguin[rotate = -30, scale=0.6, shift={([xshift=5.5cm, yshift=1.5cm]frame.north)}]
%             \end{tcbinvclipframe}
%           }
%   }

% % https://askubuntu.com/questions/706186/difference-between-and-in-linux-environment
% % $, #, % symbols indicate the user account type you are logged in to.
% %  *   Dollar sign ($) means you are a normal user.
% %  *   hash (#) means you are the system administrator (root).
% %  *   In the C shell, the prompt ends with a percentage sign (%).  

% \pgfkeys{/ubuntu/.cd,
%   user/.code={\gdef\MyUbuntuUser{#1}},user={},
%   host/.code={\gdef\MyUbuntuHost{#1}},host={},
%   color/.code={\gdef\MyUbuntuColor{#1}},color=white,
%   prompt char/.code={\gdef\MyUbuntuPromptChar{#1}},
%   prompt char=\#,
%   root/.style={user=root, host=ubuntu, color=lime, prompt char=\#},
%   bob/.style={user=bob, host=remotehost, color=cyan},
%   cat/.style={user={schroedingers\char`_cat}, host=burrow, color=magenta, prompt char=\$},
%   raspi/.style={user=pi, host=raspi, color=magenta, prompt char=\#}
% }
% \newcommand{\SU}[1]{\pgfkeys{/ubuntu/.cd,#1}%
%   \gdef\MyUbuntuPrompt{%
%     \textcolor{\MyUbuntuColor}{%
%       \small\ttfamily\bfseries \MyUbuntuUser@\MyUbuntuHost{%
%         \textcolor{white}:
%       }\textcolor{cyan!60}{\url{~}}{\textcolor{green}\MyUbuntuPromptChar} 
%     }
%   }
% }

% \newcommand{\StartConsole}{\gdef\MyUbuntuPrompt{}}

\SU{raspi}
\begin{ubuntu}
  whoami `\StartConsole`
  root `\SU{root}` 
  id `\StartConsole` 
  uid=0(root) gid=0(root) groups=0(root)`\SU{root}`
  hostname `\StartConsole`
  ubuntu`\SU{root}`
  ssh cat@burrow`\StartConsole`
  bob@remotehost's`\ `password:
  Linux remotehost 2.6.32-5-686 #1 SMP Sun Sep 23 09:49:36 UTC 2012 i686
  You have mail.
  Last login: Wed Oct 16 01:12:35 2012 from localhost
  `\SU{cat}`
  whoami`\StartConsole`
  schroedingers_cat`\SU{cat}`
\end{ubuntu}

\SU{raspi}
\begin{ubuntu}
  `\StartConsole`
  In integer abs()
  Absolute value of -10: 10 
  In long abs()
  Absolute value of -10L: 10
  In double abs() 
  Absolute value of -10.01: 10.01
\end{ubuntu}

% \SU{raspi}
% \begin{ubuntu}
%   `\StartConsole`
%   ahoju
% \end{ubuntu}
ahoj
\end{document}
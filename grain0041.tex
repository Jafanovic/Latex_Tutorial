
\documentclass{scrbook}
\usepackage{luakoma}
\usepackage{lualayout}
\usepackage{luamath} 

\begin{document}
\begin{subequations}\label{aes:eq036}
  \begin{align}
    A(jf)   &= \dfrac{A_0}{1 + \dfrac{jf}{f_0}},                  \label{aes:eq036a}  \\
    \shortintertext{resp.}   
    \dfrac{1}{A(jf)} &=  \dfrac{1}{A_0} + j\dfrac{f}{f_T}.        \label{aes:eq036b}  \\
    \shortintertext{\(f_0\) je \emph{dominantní frekvence}. Příslušná standardní amplitudová 
                    a fázová frekvenční charakteristika zesílení,}
    \abs{A} &= \dfrac{A_0}{\sqrt{1 + \frac{f}{f_0}}}, \quad 
              \arg{A} = - \arctan\dfrac{f}{f_0},                  \label{aes:eq036c}  \\ 
    \shortintertext{je vynesena na obr. Tři vyznačené údaje \(A_0\), 
                    \(f_0\) a \(f_T\) jsou vázány vztahem\protect\footnotemark[6]}
    f_T &= f_0A_0                                                 \label{aes:eq036d}
  \end{align}
\end{subequations}
\footnotetext[6]{Přesně \(f_T = f_0\sqrt{A_0^2-1} = f_0A_0\sqrt{1 - \dfrac{1}{A_0^2}}\).}

\begin{figure}[ht!]  %\ref{fyz:fig133}
  \centering
  \includegraphics[width=0.95\linewidth]{example-image-b}
  \caption{Standardní diagram barevnosti (chromatický diagram \textbf{CIE
   1931}\protect\footnotemark[5]) (\cite[s.~468]{Feynman01})}
  \label{fyz:fig133}
\end{figure}

\footnotetext[5]{Je jeden z prvních matematicky definovaných standardů}

\begin{equation}\label{fyz:eq252}
  \mathrm{grad}\,T =\nabla T = \left(\pder{T}{x},\, \pder{T}{y},\,\pder{T}{z} \right)
  \protect\footnotemark[4]
\end{equation}
\footnotetext[4]{V naší symbolice představuje výraz \((a, b, c)\) vektor se složkami \(a\),
\(b\), \(c\). Použijeme-li jednotkové vektory $\vec{i},\,\vec{j},\,\vec{k}$, můžeme psát
$$\mathrm{grad}\,T=\nabla T = \vec{i}\pder{T}{x} + \vec{j}\pder{T}{y} + \vec{k}\pder{T}{z}$$}
Použitím této nové symboliky se můžeme pokusit rovnost 
přepsat na kompaktnější tvar

\begin{mdframed}[style=mdnote]
  \begin{note}
    Výrokům se přiřazují pravdivostní hodnoty výroku takto: Je-li výrok pravdivý, přiřazuje se
    mu pravdivostní hodnota pravda označovaná symbolem (číslicí)
    \num{1}\protect\footnotemark[2]. Je-li výrok nepravdivý, přiřazuje se mu pravdivostní
    hodnota nepravda označovaná symbolem (číslicí) \num{0}. O daném výroku se pak stručně
    říká, že má pravdivostní hodnotu \num{1}\protect\footnotemark[1], resp. \num{0}. 
  \end{note}
  \footnotetext[1]{v literatuře se lze setkat též s jejím označením \(P\)}
  \footnotetext[2]{v literatuře se lze setkat též s jejím označením \(N\)}
\end{mdframed}

\begin{table}[ht!]
  \centering
  \begin{tabular}{c|ccc}
    \rowcolor[HTML]{000000} 
    \multicolumn{1}{c}{\cellcolor[HTML]{000000}
      {\color[HTML]{FFFFFF} \textbf{VSWR}}}      & 
      {\color[HTML]{FFFFFF} \textbf{\(\abs{\Gamma}\)}} & 
    \multicolumn{2}{c}{\cellcolor[HTML]{000000}
      {\color[HTML]{FFFFFF} \textbf{Odražený výkon}}}            \\ 
    \rowcolor[HTML]{000000}{\color[HTML]{FFFFFF} }           & 
                           {\color[HTML]{FFFFFF} \(\abs{s_{11}}\)}  & 
                           {\color[HTML]{FFFFFF} (\%)}       & 
                           {\color[HTML]{FFFFFF} (dB)}           \\
     1.0 & 0.000 &  0.0 & \(\infty\)  \\
     1.5 & 0.200 &  4.0 & 14.0  \\
     2.0 & 0.333 & 11.1 & 9.55  \\
     2.5 & 0.429 & 18.4 & 7.36  \\
     3.0 & 0.500 & 25.0 & 6.00  \\
     3.5 & 0.556 & 30.9 & 5.10  \\
     4.0 & 0.600 & 36.0 & 4.44  \\
     5.0 & 0.667 & 44.0 & 3.52  \\
     6.0 & 0.714 & 51.0 & 2.92  \\
     7.0 & 0.750 & 56.3 & 2.50  \\
     8.0 & 0.778 & 60.5 & 2.18  \\
     9.0 & 0.800 & 64.0 & 1.94  \\
    10.0 & 0.818 & 66.9 & 1.74  \\
    15.0 & 0.875 & 76.6 & 1.16  \\
    20.0 & 0.905 & 81.9 & 0.87  \\
    50.0 & 0.961 & 92.3 & 0.35  \\ \cline{1-4}
    \hline
  \end{tabular}
  \caption{Demonstrace vzájemného vztahu mezi poměrem stojatých vln \texttt{VSWR}, amplitudy 
           koeficientu odrazu \(\Gamma\) a velikostí odraženého výkonu\protect\footnotemark[3] 
           vyjádřeného v procentech a decibelech.}
  \label{fig_RA:VSWRGammaTable}
\end{table}

\footnotetext[3]{Return loss}

%!!! nepoužívat footnote v equation 
\begin{equation}\label{fyz:eq473}
  Z(\text{indukčnost}) = Z_L = i\omega L.\footnote{\(Z_L\) nazýváme induktance}
\end{equation}

komplexní číslo \(3 + \imath1 +\pi ∈\) \(\lim\limits_{x\to+\infty}f(x) = +\infty\) 

Je dána funkce \(f: f(x) = x^3\). Sledujme její chování, když hodnoty argumentu \(x\) budou vzrůstat
nade všechny meze \(x \to + \infty\) (viz obr. \ref{mai:fig020}). Můžeme pozorovat, že i funkční
hodnoty budou přitom neomezeně růst. V tomto případě píšeme \(\lim\limits_{x\to+\infty}f(x) =
+\infty\) nebo \(f(x)\to+\infty\) pro  \(x\to+\infty\) a říkáme, že funcke \(f\) má v bodě
\(+\infty\) limitu \(+\infty\)

Nechť \(\varphi(x) = x^2\), \(a = 1\). Potom \[f(x) = \dfrac{x^2 - 1}{x - 1}\]. Pro \(x \neq 1\)
je hodnota funkce \(f\) rovna 
\begin{equation*}
  f(x) = \dfrac{(x + 1)\cancel{(x - 1 )}}{\cancel{(x - 1)}} = x + 1. 
\end{equation*}
\begin{equation*}
  \lim\limits{x\to1}f(x)=2    \quad\text{nebo}\quad 
  f(x) \to 2                   \text{ pro }  x\to 1.
\end{equation*}

\end{document}
% https://tex.stackexchange.com/questions/458326/how-to-set-exact-margins-in-koma-script-report
% https://tex.stackexchange.com/questions/23314/how-to-remove-1-inch-paddings-without-using-hoffset-and-voffset

\documentclass[
  paper=157.794mm:210.392mm,
  DIV=36,
  headinclude=false
]{scrbook}
 
\usepackage{helvet}
\usepackage{blindtext}

\usepackage{layouts}  % https://tex.stackexchange.com/questions/183172/koma-script-typearea-and-div
\usepackage{layout}
\usepackage[headsepline]{scrlayer-scrpage}
 
\pagestyle{scrheadings}
\setkomafont{pageheadfoot}{\normalfont\normalcolor\itshape\small}
\setkomafont{pagenumber}{\normalfont\bfseries}
\ihead{\pagemark}                         % scrguien page 255
\ofoot[]{}                                % remove pagenumber from outer foot
\cfoot[]{}                                % remove pagenumber from centre foot
 
% \setlength{\headsep}{0mm}
% \setlength{\textheight}{210mm}
% \setlength{\textwidth}{157mm}

% \setlength{\headheight}{9.6pt}

% Margins
% \setlength{\oddsidemargin}{-72.27pt}  % 1in
% \setlength{\topmargin}{-72.27pt}
% \setlength{\evensidemargin}{-72.27pt}

% \setlength{\hoffset}{-25.4mm}
% \setlength{\voffset}{0mm}
 
 
\begin{document}
  \printinunitsof{mm}{\pagevalues}

  \verb|\marginparwidth|: \printinunitsof{mm}\prntlen{\marginparwidth}

  \pagediagram
  \chapter{layout}
  \layout

  % \chapter{Moje Kapitola}

  Kromě toho, \emph{oxid uhelnatý} není zcela uspokojen. Je možné, aby k sobě připoutal další atom
  kyslíku a tak dostaneme mnohem složitější reakci, ve které se kyslík spojuje s uhlíkem a současně
  dochází ke srážce s molekulou oxidu uhelnatého. Kyslíkový atom se připojí k CO a v konečném
  důsledku vytvoří molekulu složenou z jednoho uhlíku a dvou kyslíků. Tato molekula má označení CO2
  a nazývá se \emph{oxid uhličitý}. Spalujeme-li uhlík ve velmi malém množství kyslíku a reakce
  probíhá velmi rychle (např. v motoru automobilu, kde je výbuch tak rychlý, že se nestačí vytvořit
  oxid uhličitý), vzniká velké množství oxidu uhelnatého. V mnoha takových přeskupeních atomů se
  uvolňuje velké množství energie, vznikají výbuchy, plamen apod., podle druhu reakce. Chemici
  studovali takové seskupení atomů a zjistili, že každá látka představuje určitý druh
  \emph{uspořádání atomů}.

  K objasnění této myšlenky si zvolme jiný příklad. Ocitneme-li se na louce rozkvetlé fialkami,
  víme, co je to za „vůni“. Je to určitý druh molekul nebo seskupení atomů, které se dostalo do
  našeho nosu. Jak se nám to stalo? To je dost jednoduché! Jestliže vůně je jistý druh molekul, tím
  nejrozmanitějším způsobem poletujících a srážejících se ve vzduchu, pak se může náhodou dostat i
  do nosu. Tyto molekuly se určitě nesnažily dostat právě do našeho nosu. Jsou jen bezmocnou částí
  strkajícího se zástupu molekul, jehož kousek se na svém bezcílném putování dostal do našeho nosu.
  
  
  Chemici mohou i takové zvláštní molekuly, jako je vůně fialek, podrobit analýze a říci nám
  \emph{přesné uspořádání} jejich atomů v prostoru. Víme, že molekula oxidu uhličitého je
  \emph{přímá a symetrická}: (lze to snadno zjistit i fyzikálními metodami). I pro mnohem složitější
  seskupení atomů, jako jsou ty, se kterými pracuje chemie, můžeme zdlouhavým, pozoruhodným
  procesem, připomínajícím práci detektiva, zjistit tvar seskupení. Obr. znázorňuje vzduch v
  blízkosti fialky: ve vzduchu opět nalézáme dusík, kyslík a vodní páru. (Odkud se vzala vodní pára?
  Fialka je vlhká, protože všechny rostliny odpařují vodu.) Vidíme však i monstrum složené z
  uhlíkových, vodíkových a kyslíkových atomů, které vytvořily zcela určité, zvláštní seskupení. Je
  to mnohem složitější seskupení než v případě oxidu uhličitého. Naneštěstí do obrázku nemůžeme
  zakreslit všechno, co o něm po chemické stránce víme, neboť seskupení všech atomů je trojrozměrné,
  zatímco náš obrázek je pouze dvojrozměrný. Šest uhlíků vytváří ne plochý, ale zvrásněný prstenec.
  Všechny úhly a vzdálenosti známe. Chemický vzorec je jen obrázkem takové molekuly. Když chemik
  napíše vzorec na tabuli, snaží se nakreslit dvojrozměrný obraz molekuly. Například, vidíme
  prstenec šesti uhlíků a na jednom konci visící řetěz uhlíků, na něm kyslík druhý od konce, tři
  vodíky vázané na tento uhlík, dva uhlíky a tři vodíky vázané nahoře atd.

  Kromě toho, \emph{oxid uhelnatý} není zcela uspokojen. Je možné, aby k sobě připoutal další atom
  kyslíku a tak dostaneme mnohem složitější reakci, ve které se kyslík spojuje s uhlíkem a současně
  dochází ke srážce s molekulou oxidu uhelnatého. Kyslíkový atom se připojí k CO a v konečném
  důsledku vytvoří molekulu složenou z jednoho uhlíku a dvou kyslíků. Tato molekula má označení CO2
  a nazývá se \emph{oxid uhličitý}. Spalujeme-li uhlík ve velmi malém množství kyslíku a reakce
  probíhá velmi rychle (např. v motoru automobilu, kde je výbuch tak rychlý, že se nestačí vytvořit
  oxid uhličitý), vzniká velké množství oxidu uhelnatého. V mnoha takových přeskupeních atomů se
  uvolňuje velké množství energie, vznikají výbuchy, plamen apod., podle druhu reakce. Chemici
  studovali takové seskupení atomů a zjistili, že každá látka představuje určitý druh
  \emph{uspořádání atomů}.

  Chemici mohou i takové zvláštní molekuly, jako je vůně fialek, podrobit analýze a říci nám
  \emph{přesné uspořádání} jejich atomů v prostoru. Víme, že molekula oxidu uhličitého je
  \emph{přímá a symetrická}: (lze to snadno zjistit i fyzikálními metodami). I pro mnohem složitější
  seskupení atomů, jako jsou ty, se kterými pracuje chemie, můžeme zdlouhavým, pozoruhodným
  procesem, připomínajícím práci detektiva, zjistit tvar seskupení. Obr. znázorňuje vzduch v
  blízkosti fialky: ve vzduchu opět nalézáme dusík, kyslík a vodní páru. (Odkud se vzala vodní pára?
  Fialka je vlhká, protože všechny rostliny odpařují vodu.) Vidíme však i monstrum složené z
  uhlíkových, vodíkových a kyslíkových atomů, které vytvořily zcela určité, zvláštní seskupení. Je
  to mnohem složitější seskupení než v případě oxidu uhličitého. Naneštěstí do obrázku nemůžeme
  zakreslit všechno, co o něm po chemické stránce víme, neboť seskupení všech atomů je trojrozměrné,
  zatímco náš obrázek je pouze dvojrozměrný. Šest uhlíků vytváří ne plochý, ale zvrásněný prstenec.
  Všechny úhly a vzdálenosti známe. Chemický vzorec je jen obrázkem takové molekuly. Když chemik
  napíše vzorec na tabuli, snaží se nakreslit dvojrozměrný obraz molekuly. Například, vidíme
  prstenec šesti uhlíků a na jednom konci visící řetěz uhlíků, na něm kyslík druhý od konce, tři
  vodíky vázané na tento uhlík, dva uhlíky a tři vodíky vázané nahoře atd.

  \blindtext[5]
 

 
\end{document}